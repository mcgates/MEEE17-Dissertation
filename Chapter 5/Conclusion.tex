This dissertation has explored the impact of renewable energy integration on DC grid stability and efficiency. Through a comprehensive analysis of residential energy management and control systems (REMCS), this research has demonstrated that such systems can significantly enhance the utilization of renewable resources while maintaining grid stability.\par

\section{Key findings}
\subsection{Efficiency in Energy Utilisation}
The EMS improved system efficiency by adjusting to real-time consumption patterns. This prevents overloading and excessive discharge of battery storage, thus optimizing energy usage and extending battery life.\par
\subsection{Power Sharing}
The power share feature, detailed in Figure 4.44, allows for surplus energy from fully charged local battery storage to be redirected to the grid. This not only aids in rapid recharging but also in supplying connected loads, enhancing overall energy distribution efficiency.\par
\subsection{Grid Dependence and Battery Capacity}
The laboratory results for Unit 2, as outlined in Table 4.10, reveal that grid reliance for battery charging is crucial due to limited battery capacity and lack of PV modules. The control system’s ability to disconnect loads to maintain grid stability highlights the importance of strategic energy management in scenarios with varying load demands.\par
\subsection{Comparative analysis of Supply and Demand}
Figure 4.45 illustrates the comparative performance of supply capacity versus load demand. Prioritizing grid usage over local storage allows for extended operation of local loads, although the rapid decrease in the state of charge (SoC) necessitates frequent switching to grid power.\par
This research contributes to the field by providing empirical evidence supporting the efficiency and reliability of REMCS in renewable energy systems. Future work could focus on integrating with energy utilities and integrating more advanced predictive algorithms to further enhance energy management. Overall, the findings underscore the potential of dynamic energy management systems to play a pivotal role in the transition towards more sustainable and resilient energy infrastructures.\par

\section{references}
