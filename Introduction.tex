\section{Statement of research problem}
The utilization of renewable energy resources (RES) is on the rise, as many countries are making efforts to transition away from non-renewable energy sources due to their contribution to carbon emissions in the atmosphere. Microgrids combine distributed energy resources, energy storage, and load management. They have found applications in electrifying off-grid rural villages and remote regions. DC Microgrids (DCMG) have gained widespread use in aerospace, automotive, marine, and other industries. They serve as crucial components for integrating DERs (Distributed Energy Resources), especially since most renewable energy sources generate DC power. A nano grid, on the other hand, is designed for power distribution in a single house or small building [1]. A DC nano grid typically includes DERs, MPPT (Maximum Power Point Tracking), and ESS (Energy Storage System).\\
The use of renewable energy resources (RES) continues to grow. Numerous nations are seeking to transition from non-renewable energy sources as a result of the carbon emissions released into the atmosphere. A microgrid combines distributed energy resources, energy storage, and load. They have been used in the electrification of off-grid rural villages and remote areas. DC Microgrids (DCMG) have been widely used in aerospace, automotive, marine, and other industries. They are the critical components in integrating DERs since most renewable energies produced are in DC form. A nano grid is a power distribution system for a single house/small building. A DC nano grid consists of a DER, MPPT, and ESS.